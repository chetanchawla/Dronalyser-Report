\chapter{Related Work}
Remote sensing \cite{two} with the unmanned aerial systems (UAS) with multispectral imagery dates back to 2008 \cite{eight-icuas} and earlier where results were generated using traditional approaches of using high cost wide band multi-spectral cameras with application of physical filtering. The field is now emerging as an application which systematically applies monitoring of vegetation and crops to enable smart decisions for the farmers in the agricultural activities. The field has a  lot of dependence on the satellite imagery for collection of hyperspectral imagery. However, recently the paradigm has shifted to cheaper UAS based solution which is able to gather large amounts of raw data to process for an be used to monitor crop health in more than one way such as water level concentration identification \cite{two-remotev2}, vigor analysis \cite{three-remotev2}, biomass estimation and disease monitoring using algorithmic analysis based on data from hyperspectral and thermal imagery as well as using machine learning applications of clustering (unsupervised) and supervised by collection of large amounts of labelled data of particular fields.  
\\
\\
Both multispectral analysis and the hyperspectral analysis requires acquisition of data aerially, at a fixed position in altitude and coverage of the entire crop field, using low cost passive imagery sensors such as the normal visible light camera (RGB camera), Near Infra Red (NoIR camera) and relatively expensive hyperspectral cameras. The sensors are distinguished by the bands (channels) of electromagnetic spectra reflected by the crop and their corresponding widths. The spatial distribution of the energy reflected by the plants to these sensors differ in the bands of frequency which points to the variation of plant health spatially in the crop field. Each spectrum of energy reaching the sensors are captured based on the specification of the cameras. The multispectral imagery ranges from wide bands 5 to 12 represented as pixels. The hyperspectral imagery has hundreds or even thousands of bands present as narrow width bands (5-20 nm each). The multispectral imagery we use is an adjusted version of the expensive multispectral imagery which only captures the most significant band for chlorophyll reflectance, i.e., the Near IR band. The solution can be used by removing the IR filter present in RGB cameras for the camera to capture the Near IR energy spectrum reflected by the crops as well. This can be isolated from the RGB bands by using a band filter for one of the three visible bands. We used a Red light filter for the same.
\begin{figure}[H]
    \centering
    \includegraphics[width=0.7\linewidth]{SummerInterReport/project/Images-Major/em_spectra.png}
    \caption{Spectrum representation including: (A) Multispectral example, with 5 wide bands; and (B) Hyperspectral example consisting of several narrow bands that \cite{fourteen-remotev2}).}
    \label{fig:Concise Flow}
\end{figure}

\\
\\

Both Multispectral and Hyperspectral imagery can be a boon to the agricultural sector over its analysis range to asses even the smallest details such as telling the elemental defciency in the spatial context of the crop to enable precision farming for the individuals. Also, productivity and stress indicators in both agricultural systems can be assessed through photosynthetic light use efficiency quantification, which can be obtained by measuring the Photochemical Reflectance Index (PRI) relying on narrow band absorbance of xanthophyll pigments at 531 and 570nm \cite{fifteen-remotev2}.  However commercial hyperspectral cameras without the acquisition systems and the data processing systems cost upwards of 6000\$, deeming it infeasible in Indian conditions where the infographic of a higher detail would cost leaps more to the farmer as well as would be non-usable due to the higher levels of complexity in data, incomprehensible by a large majority (62\% uneducate as cited by \href{https://timesofindia.indiatimes.com/india/62-farmers-cannot-meet-educational-needs-Survey/articleshow/12049496.cms}{2012 reports of Times Of India}). Commercial multispectral systems costs up to 5000\$ or Rs 3.5 lac however we successfully managed to reduce the expenditure in our system for data acquisition as well as processing to Rs 3000 or 50\$ (using a raspberry Pi 3B+, supporting expansion in ML domains as well fr clustering the hyperspectral imagery in future) and a NoIR camera.
\\
\\
Furthermore, we have used the APM 2.8 flight controllers and tested the algorithms of path planning with Primus V3R in constrained environments as well as CC3D mini, which are cheaper than the commonly used PixHawk or Pixhack flight controllers.



