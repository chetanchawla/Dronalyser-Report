\section{Evaluation}

\subsection{DRIZY - VISION}
\paragraph{}This section addresses the question of how well Drizy can detect pedestrians keeping sufficient reaction time for the driver.
\subsubsection{Accuracy}
\paragraph{}Accuracy of a pedestrian detector is generally determined on a frame-by-frame basis. However, Drizy-VISION, a warning system installed in a car, is not affected if all the pedestrians are detected or correctly localized in the frame, instead it validate if an alert is generated when pedestrians walk in front of the car or not.
% We determine the accuracy of Drizy - Vision when a pedestrian detector is used as a warning system in a car. Instead of determining if each pedestrian is accurately localized on a frame-by-frame basis, we validate if an alert is generated when pedestrians walk in front of the car or not.\\
\paragraph{}Using this method, various parameters were evaluated on a video dataset collected on Indian roads with 5500 positives(pedestrians) and 4500 negatives (non-pedestrian/background). All the algorithms were benchmarked on Raspberry Pi 3. Figure \ref{fig:eval2} shows that state of the art techniques like faster RCNN and YOLO offer high precision rates (Equation \ref{eq1}) of 98 to 99 percent with an fps of 0.8 to 1 fps. HOG+SVM classifier (proposed system) provides a speedup 10x over deep learning at a precision of 80 percent.
\paragraph{}Figure 2 shows that HOG+SVM classifier performs better over HAAR with a gain of 0.4 percent recall (Equation \ref{eq2}) at 0.1 FPPI (false positives per image).
\begin{equation}
    Precision = \frac{True Positives}{True Positives + False Negatives}
    \label{eq1}
\end{equation}\\

\begin{equation}
    Recall = \frac{True Positives}{True Positives + False Negatives}
    \label{eq2}
\end{equation}\\

\begin{figure}
\centering
{
\includegraphics[scale=0.85]{project/images/eval2.pdf}
\caption{\textbf{Left: Recall vs FPPI for pedestrian classifiers at varying thresholds. Right: Precision vs frames processed per second for various detection techniques.}}
\label{fig:eval2}
}
\end{figure}
\subsubsection{Reaction Time}
\paragraph{}An alert system should not only detect pedestrians accurately, but also give sufficient to the driver to apply brakes. Here we calculate average reaction time available to the driver from the time of alert to the time of collision with pedestrian.
\paragraph{}The pedestrian detection module is capable of successful detection upto a range of 18 to 23m at varied lighting conditions.A set of experiment was carried out at IIT Delhi campus roads with 50 test drives in a car. The car was maintained at constant speed of 20, 30, 40 and 50km/hr for 10 test drives each. For every drive the time taken to reach the pedestrian after alert reception by Drizy-VISION was noted down. Figure this this this shows that the system warns the driver 4-5 sec prior to approaching a pedestrian ahead at 10-40 km/hr speed limit.
\begin{figure}[hbtp]
\centering
{
\includegraphics[scale=0.6]{project/images/eval1.pdf}
\caption{\textbf{(a) Available Reaction time before collision with pedestrians and vehicles. It shows maximum and minimum values of time obtained during test-runs. (b) Comparative analysis of a system that gives alerts to every vehicle in the accident prone area with DRIZY that predicts collisions before giving alerts. The values were obtained using simulation and spawning of vehicles with random attributes in an accident blackspot.}}
\label{fig:eval1}
}
\end{figure}
\subsection{Drizy - COM}
\paragraph{}This section evaluates Drizy-COM on the basis of the following parameters:\\
(i) available time for driver to apply brake before collision.\\
(ii) frequency of alerts generated by DRIZY. We compare DRIZY with an "Always On" system that gives alerts to every vehicle in the accident blackspot.\\
(iii) accuracy of localization of vehicles. 

\subsubsection{Reaction Time}
\paragraph{}We simulate the same experiment as in Drizy-VISION to analyze reaction time for Drizy-COM at road intersections. We take 50 test drives with 2 vehicles approaching a road intersection from two different roads at varying speed. We analyze the available reaction time for the driver after the collision alert and collision prediction behaviour of the DRIZY app.
Figure \ref{fig:eval1} (a) shows that for speeds limits between 20-25 kmph the system generates alerts with available reaction time >20s and for speed limits reaching 40-45 kmph, it is found to be close to 8s.

\begin{figure}[hbtp]
\centering
{
\includegraphics[width=\linewidth, height=6cm]{project/images/gpserror.png}
\caption{\textbf{Box plot showing deviation in GPS vaues for different mobile phones}
\label{fig:gps}
}}
\end{figure}
\subsubsection{Comparison with "Always On System"}
\paragraph{}In order to study the frequency of collision alerts in different traffic densities, we simulate an experiment. We generate synthetic data with uniform distribution for \lq{N}\rq vehicles which includes vehicle distance from intersection ranging between 0-400m, road number from 0-4, vehicle speed in range 1-36kmph and direction of motion of vehicle. This data is then accessed by the server to predict collisions. Figure \ref{fig:eval1} (b) shows the comparison of DRIZY with an Always On system at low, medium and high density traffic levels. The proposed system reduces the redundant warnings by 80\%, 50\% and 10\% in low, medium and high traffic density, respectively.

\subsubsection{Network Attributes}
\paragraph{}Since the assistance feature is dependent on the network speed, we show the accuracy of GPS data and latencies associated with upload, download and server processing over 4G and WiFi networks. Figure \ref{fig:gps} shows a box plot of GPS errors experimented on different mobile phones. The threshold line shows the maximum tolerable limit in deviation that does not interfere in assistance system. It can be observed that mobile phones launched after 2014 are within the tolerable limit.
\paragraph{}Table 1 shows the latencies of both 4G and WiFi netwoks for different processes. It can be observed that the throughput of the system is not dependent on the type of network or bandwidth of network, it just requires a network connectivity. This behaviour of Drizy-COM is due to the small packet size of the vehicle data that is being uploaded.

\begin{table}[h]
\center
\label{tab1}
 \begin{tabular}{| c | c | c |} 
 \hline
 \multirow{} & \multicolumn{2}{c|}{Time (milliseconds)}   \\\cline{2-3} 
 \textbf{Process} & \textbf{4G} & \textbf{Wifi}\\  
 \hline
 Upload & 258 & 229\\ 
 \hline
 Download & 746 & 678
 \\
 \hline 
 Server processing & 30 & 30
 \\
 \hline
 \end{tabular}
 \caption{Average latency of Drizy-COM for different types of networks.}
\end{table}

\newpage