\begin{center}
\thispagestyle{empty}
\vspace{2cm}
\LARGE{\textbf{ABSTRACT}}\\[1.0cm]
\end{center}
\thispagestyle{empty}
\large{\paragraph{}
Agriculture is still the primary means of earning for a major sect of our society and is responsible to uphold our ever-growing population. According to 2015 reports by World Bank Collection of development indicators, around 60.45\% of the land in India is agricultural land. However, large proportion of the produce goes to waste due to lack of proper and timely analysis of the crops and the field due to over-dependence on ineffective primitive methods. There is an annual loss of 15-20\% of the crop produce which accounts for around 397 thousand square kilometers of land wasted annually. Crop loss is generally diagnosed by the symptoms developed by pest and diseases, low nutrition and lack of proper irrigation practices. While manual diagnosis has a very limited scope in identifying the damaged plant parts, recognizing the pattern of pest/disease distribution is almost close to impossible in a large area for the general farmers. This rules out the manual monet analysis of the crop fields by the farmer for pests \& diseases and for assessing the plant health. Similarly, imperfect irrigation practices lead to either a large loss of water resources (due to over-irrigation) or patches of the crop lack water (due to under-irrigation or uneven irrigation), thereby decreasing the efficiency. 
We thereby propose a drone based monitoring and visualization service for the farmers to assess the plant health of the crop fields using effective and cost efficient multispectral imagery system, on-board processing and an easy to use interface. The interface is provided through an android application. The application serves as an intermediary between the local drone dispatch station and the crop field. The drone, on reaching the requested location, analyses the field using an on-board camera capturing field aerial imagery to exploit the visible blue and near-IR spectra (multi-spectral imagery) to calculate chlorophyll presence as a metric. We use this metric to assess the crop health and present the infographic to the farmer.
%and a hyper-spectral camera for elemental analysis. The crop-field analysis is then converted as an infographic and displayed to the user for intelligent farming.
%
}
