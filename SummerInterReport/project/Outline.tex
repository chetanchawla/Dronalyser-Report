\begin{center}
\thispagestyle{empty}
\vspace{2cm}
\LARGE{\textbf{ABSTRACT}}\\[1.0cm]
\end{center}
\thispagestyle{empty}
\large{\paragraph{}
India is a big land-mass of valuable resources sought after by the whole world for times unknown. However the crop fields in India are the actual gold of the resources one could have. According to 2015 reports by World Bank Collection of development indicators, around 60.45\% of the land in India is agricultural land. However, we still are very inefficient in the usage of the crop and a large proportion of the crop goes to waste due to loss of knowledge about the crop and the field. There is an annual loss of 15-20\% of the crop which accounts for around 397 thousand square kilometers of wasted land annually. Crop loss is generally diagnosed by the symptoms developed by pest and diseases, low nutrition and lack of proper irrigation practices. While manual diagnosis has a very limited scope in identifying the damaged plant parts, recognizing the pattern of pest/disease distribution is almost close to impossible in a large area for the general farmers. This rules out the manual monet analysis of the crop fields by the farmer for pests and diseases, which even after being discovered through skim chances, leaves an improbable guess as to what nutrients are deficient in the crop. Similarly, with imperfect irrigation practices, either a large loss of water resource is resulted or patches of the crop lack water, thereby decreasing the efficiency. 
We thereby propose a drone based monitoring and visualization service for the farmers using an easy to use interface of android application. The application serves as an intermediary between the local Drone dispatch station and the crop field, which thereafter analyses the field using 2 on-board cameras capturing field aerial imagery, combining them to form the orthomosaic, in rgb and IR spectra (multispectra imagery) and a thermal sensor array for water stress level analysis using thermal coefficient of water.
%and a hyper-spectral camera for elemental analysis. The crop-field analysis is then converted as an infographic and displayed to the user for intelligent farming.
%
}
