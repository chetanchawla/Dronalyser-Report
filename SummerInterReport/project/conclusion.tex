\chapter{Conclusion and Future Scope}
\section{Conclusion}
A complete drone end-to-end service was designed for the use of multispectral imagery. We developed a bilingual android application with easy to use interface for the farmers to call drones and view infographic information of the crop fields, test drones with different flight controllers and a data acquisition and processing system on-board the drone. We figured out the best board to use in the scenario as the cheapest alternative is ArduPilot APM2.8 with GPS enabled. The board was fixed on a custom made 250 mm carbon fibre chassis. The data acquisition system costed around 50\$ making it one of the cheapest multispectral imagery set-up. We tested the drone in open environments and assessed the crop health as well. Providing it as a service increases the feasibility of the proposed solution in the Indian market.
\section{Future Scope}
The future work involves improvisations and optimizations to the existing framework of multispectral imagery with addition of designated Thermal Sensors like AMG833 to calculate the water stress levels and Hyper-spectral cameras like Rikola Hyperspectral cameras along with Machine Learning Clustering algorithms to find out elemental deficiencies and potential crop diseases.
\\
We also tend to incorporate the satellite modules for remote sensing modularity added in the mainframe. Satellite imagery can also help us assess the temporal features of the crops which act as an important delimiter in the yield and overall health assessment of the soil.
%add 'em

% Our work is future proof and cost effective
% User independent
% Cost-effective
% Easily deployable
% The system helps us achieve our target, of 250-million connected vehicles by 2020, using a robust framework.
% How we make a dent and why should someone buy the system 