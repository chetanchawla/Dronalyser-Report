\newpage
\begin{center}
\thispagestyle{empty}
\vspace{2cm}
\LARGE{\textbf{Conclusion and Future Scope}}\\[1.0cm]
\end{center}

\paragraph{}
We started this project with the aim to solve the social problem of road safety in India. It took us no time to realize that it's a long way down the road for driver assistance systems here. To fill these gaps fast, different companies must join hands to create a collaborative system just like it happened a decade ago to create maps. We contributed to this field by implementing Drizy, which is based on the concept of connected cars and computer vision.

\paragraph{}In order to achieve desired accuracies in detecting pedestrians in video input from a dashboard camera as in Drizy-VISION, we took up the challenge of compiling a Pedestrian Detection Video Dataset recorded on Indian roads to train deep learning models. Realising that the existent video annotation toolboxes did not fill in our needs, we also launched our very own customised toolbox - labelVDOS.

\paragraph{}The future prospects of our project include determining and improving the accuracies of existent object detection models like Faster RCNN by fine tuning them on the Indian Pedestrian Detection Dataset. We also wish to extend our work for Pedestrian Trajectory Estimation. 
% \paragraph{}The system finds its utility even in bad weather conditions since prediction of collisions at intersection points is based on vehicle-to-cloud communication. On the other hand, the functionality of Drizy-VISION is aimed specifically to aid distracted drivers. Although Drizy has currently been prototyped for two types of assistance features only, the applicability of the system can be expanded using the same framework.
% \paragraph{}Initial stages in developing Drizy-COM involved experiments on setting up peer-to-peer networks over ZigBee protocols. However, this limited communication between vehicles in the range of 60-70 metres. These networks would also be limited to vehicles on a straight road, whereas at bends, obstacles like buildings would hinder communication. Therefore, communication over cloud was chosen due to its robustness. Latency issues related to data uploading, cloud-computing and data downloading were dealt with by employing the concept of edge-computing.
% \paragraph{}Results validated that a collaborative driver assistance system is feasible in low and medium traffic density. The current system design could be limited merely by computation power of raspberry pi and internet connectivity. It is independent of internet speed due to extremely small packet of data being uploaded and downloaded. Optimizations implemented to speed up computation on embedded systems could make such systems cost-effective. Thus, such systems are easily deploy-able and will be the future of driver assistance systems in developing countries with complex road conditions. They could be a solution to improving road safety and achieving the target of lowering road accidents and fatalities to half by 2020. 
% In future, we plan to design a peer-to-peer network of cars which can help overcome the dependency of the system on internet connection which is still a matter of concern in certain areas in developing countries. In order to increase the utility of the system, we also plan to add a feature that helps vehicles on a road to coordinate and clear out the way for ambulances to pass, thereby creating a "Green Corridor". \\
%Not sure of this above statement
% This can be our future work. But i doubt if we will actually do it. Same. even the green corridor thing . Lets just add then.Okay. Also, i have sent the TODO list on hangouts.Oh okay. I will keep myself involved in presentations and videos for the next week. let me know if you can help me in this. Sure I . have a look at snapchat for element of hapiness. downloading snapchat lol. rehne de, insta kar diya. refreshing insta
% As discussed earlier, Raspberry Pi 3 with limited compute power was not sufficient for heavy processing involved in state-of-art computer vision algorithms based on deep learning. With a trade-off between speed and accuracy we designed Drizy which can alert the driver of impending collision with pedestrians and other vehicles.\\
% The system finds its utility in bad weather conditions when the driver has limited visibility or may not be paying attention to its environment.\\
% The major challenges we faced in the project includes limited network bandwidth and poor visibility of lane markings. However, they were both taken care of using concepts like edge computing and image segmentation.\\
% Drawbacks of Drizy includes its lack of communicating with cars which do not have Drizy installed. This may reduce the overall utility of the system. In future we plan to design a peer-to-peer network in cars which can overcome this drawback and decrease latency.
