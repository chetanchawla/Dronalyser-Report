\addcontentsline{toc}{chapter}{References}
\begin{thebibliography}{}
\bibitem{one}Andres Montes de OcaLuis, Arreola Alejandro, FloresShow, Gerardo Flores, "Low-cost multispectral imaging system for crop monitoring" in Conference: ICUAS'18 The 2018 International Conference on Unmanned Aircraft Systems at Dallas, TX, USA, May 2018

\bibitem{two}Adao, T., Hruska, J., Padua, L., Bessa, J., Peres, E., Morais, R. Sousa, J.J., " Hyperspectral Imaging: A Review on UAV-Based Sensors, Data Processing and Applications for Agriculture and Forestry." in Remote Sens. 2017, 9, 1110. 

\bibitem{eight_icuas}A. M. Jensen, M. Baumann, and Y. Chen, "Low-cost multispectral aerial imaging using autonomous runway-free small flying wing vehicles," in IGARSS 2008 - 2008 IEEE International Geoscience and Remote Sensing Symposium, vol. 5, July 2008, pp. V - 506-V - 509

\bibitem{two_remotev2}Park, S., Nolan, A., Ryu, D.; Fuentes, S., Hernandez E., Chung, H., O'Connell M., " Estimation of crop water stress in a nectarine orchard using high-resolution imagery from unmanned aerial vehicle (UAV)" In Proceedings of the 21st International Congress on Modelling and Simulation, Gold Coast, Australia, 29 November–4 December 2015; pp. 14131419. 

\bibitem{three_remotev2}Primicerio, J., Gennaro, S.F. D., Fiorillo E., Genesio, L., Lugato, E., Matese, A., Vaccari, F.P., " A flexible unmanned aerial vehicle for precision agriculture" Precis. Agric. 2012, 13, 517–523

\bibitem{fourteen_remotev2}Multispectral vs. Hyperspectral Imagery Explained. Available online: http://gisgeography.com/ multispectral-vs-hyperspectral-imagery-explained/ (accessed on 15 September 2017). 

\bibitem{fifteen_remotev2}Proctor C., He, Y., " Workflow for Building A Hyperspectral Uav: Challenges And Opportunities", ISPRS Int. Arch. Photogramm. Remote Sens. Spat. Inf. Sci. 2015, XL-1/W4, 415–419. 

\end{thebibliography}
